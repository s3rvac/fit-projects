%==============================================================================
% Encoding: utf8
% Project:  ITY - Project 1
% Author:   Petr Zemek, xzemek02@stud.fit.vutbr.cz
%==============================================================================

\documentclass[12pt,a4paper]{article}

% Packages
\usepackage[utf8]{inputenc}
\usepackage[czech]{babel}
\usepackage{times}
\usepackage[left=2cm,text={17cm, 24cm},top=3cm]{geometry}

% Commands
\newcommand\uv[1]{\quotedblbase #1\textquotedblleft}

\begin{document}

% Macros
% Now, switch on what is appropriate for czech:

% czech quotation marks
% \bq - begin quotation, \eq - end quotation
\def\bq{\mbox{\kern.1ex\protect\raisebox{-1.3ex}[0pt][0pt]{''}\kern-.1ex}}
\def\eq{\mbox{\kern-.1ex``\kern.1ex}}
%\setlanguage{\czech}

{%                                      % Begin a group for which " is active
\catcode`\"=\active                     % Make " an active character
\catcode`\@=11                          % Make @ an active character
%
%  \csdoublequoteson
%
%       This macro makes " an active character, resets the control sequence
%       \dblqu@te to L (left), and defines \dq as a replacement for ".
%
\gdef\csdoublequoteson{%                % \csdoublequoteson enables "
    \gdef"{\czechquotes}%               % Define " as \czechquotes
    \global\catcode`\"=\active%         % Make " an active character
    \global\chardef\dq=`\"%             % Double-quote char. via \dq
    \global\let\dblqu@te=L%             % Always start with a left double-quote
    }                                   % End of macro
%
%  \bq, \eq
%
%      These macros define default characters for czech left and right
%      double quotes. Czech opening quote is created from two commas with
%      kerning depending on fontdimen four parameter of current font.
%      Better solution should be specially designed character with
%      proper kernings; if you have such characters in fonts
%      (e.g. in DC-fonts), use it instead. (e.g. define
%      macros \bq and \eq e.g. \def\bq{\char"130 }
%      in your document/style file-- not in csquote.sty!)
%      Similar solution is used for czech right quote.
%
%      \cs existence test, stolen from TeXbook exercise 7.7
\def\ifundefined#1{\expandafter\ifx\csname#1\endcsname\relax }%
%
%      another macro to be more efficient in time and space
\global\chardef\f@@r=4
%
\ifundefined{bq}%
\gdef\bq{\kern-.25\fontdimen\f@@r\font,\kern-.8\fontdimen\f@@r\font,%
                \kern-.35\fontdimen\f@@r\font}%
\fi
\ifundefined{eq}%
\gdef\eq{\kern-.35\fontdimen\f@@r\font`\kern-.8\fontdimen\f@@r\font`%
                \kern-.25\fontdimen\f@@r\font}
\fi
%
% Macro \uv for other usage of \bq and \eq.
%
\ifundefined{uv}%
        \gdef\uv#1{\bq #1\eq}
\fi
%
% \testquotes macro gives warning if citation span this place
%
\gdef\testquotes{\if R\dblqu@te
        \message{Warning: You forgot right double quote!}%
        \let\dblqu@te=L\fi}
%
%  Define the macro that will be executed whenever " is encountered.
%
\gdef\czechquotes{\protect\czechqu@tes}
\gdef\czechqu@tes{%
        %  If the double-quote is the first character in a new paragraph,
        %  make sure it becomes a left double-quote.  This case can be
        %  detected by checking to see if TeX is currently in vertical mode.
        %  If so, the double-quote is at the beginning of the paragraph
        %  (since " hasn't actually generated any horizontal mode tokens
        %  yet, TeX is still in vertical mode).  If the mode is vertical,
        %  set \dblqu@te equal to L.
        %
        \ifinner\else\ifvmode\testquotes\fi\fi%
        %
        %  Now insert the appropriate left or right double-quote.
        %
        %  If \dblqu@te is L, insert an opening quote and set \dblqu@te to R.
        %
        \if L\dblqu@te\bq\global\let\dblqu@te=R%
        %
        %  Otherwise, save the current \spacefactor, insert '', set \dblqu@te
        %  to L, and reset the original \spacefactor.
        %
        \else%
           \let\xxx=\spacefactor%               % Save the \spacefactor
           \eq%                                 % Insert ending quote
           \global\let\dblqu@te=L%              % and reset \dblqu@te
           \spacefactor\xxx%                    % Reset the \spacefactor
        \fi%                                    % End of \if L\dblqu@te...
        }                                       % End of " macro
}                                               % End of group

\gdef\csdoublequotesoff{%
        \catcode`\"=12%                         % Set " back to other
        }
%
% Czech quotes are default
%
\csdoublequoteson




\section{Hladká sazba}

Hladká sazba je sazba z~jednoho stupně, druhu a řezu písma sázená na stanovenou šířku odstavce. Skládá se
z~odstavců, které obvykle začínají zarážkou, ale mohou být sázeny i bez zarážky -- rozhodující je celková
grafická úprava. Odstavce jsou ukončeny východovou řádkou.

Pokud sázíme text hladkou sazbou, nelze v~textu používat různé druhy a řezy písma, zvětšovat, či zmenšovat
velikost písma vybraných slov. Rovněž bychom neměli používat barevné zvýraznění. Hladká sazba je určena
především pro sazbu delších textů. Typicky se používá pro sazbu beletrie. Porušení konzistence sazby působí
v~textu rušivě a unavuje čtenářův zrak.

\section{Smíšená sazba}

Smíšená sazba nemá tak omezující pravidla, jako sazba hladká. Nejčastěji se klasická hladká sazba doplňuje
o~další řezy a druhy písma z~důvodu zvýraznění důležitých pojmů. Existuje důležité pravidlo:

\begin{quotation}
	Čím \texttt{více} {\Large druhů}, \textit{řezů},
	{\Huge v}{\LARGE e}{\Large l}{\large i}{\normalsize k}{\small o}{\footnotesize s}{\scriptsize t}{\tiny í}
	a barev \underline{písma} \textit{použijeme}, tím se \underline{\textbf{\textit{dokument}}} stane
	\textsl{čitelnější}. Čtenáři to \underline{vždy} \textsc{ocení}!!!
\end{quotation}

Tímto pravidlem byste se nikdy neměli řídit. Příliš časté zvýrazňování textových elementů jednak svědčí
o~amatérismu autora (jaké štěstí, že do tištěných dokumentů nelze vkládat pohyblivé obrázky) a navíc to působí
velmi rušivě. Dobře navržený dokument by měl obsahovat maximálně 4 řezy či druhy písma. Dobře navržený
dokument je decentní, ne chaotický.

Důležitým znakem profesionálně vysázených dokumentů je konzistence používání různých druhů zvýraznění.
Smíšená sazba se nejčastěji používá pro sazbu technických zpráv a vědeckých článků. Konzistentní používání
zvýrazňovačů může například znamenat, že \textbf{tučný řez} písma bude vyhrazen pouze pro klíčová slova,
\textit{skloněný řez} pouze pro doposud neznámé pojmy, a že se to nebude míchat. Skloněný řez působí méně
rušivě a používá se častěji. V~\LaTeX{u} pro jeho sázení používáme raději příkaz \verb!\emph{text}! než
\verb!\textit{text}!.

U~delších dokumentů technického či vědeckého charakteru bývá dobrým zvykem upozornit čtenáře na použitý styl
v~úvodní kapitole.

\section{České odlišnosti}

Česká sazba se oproti okolnímu světu v~některých aspektech mírně liší. Jednou z~odlišností je sazba uvozovek.
Uvozovky se v~češtině používají převážně pro zobrazení přímé řeči. V~menší míře se používají také pro
zvýraznění přezdívek a ironie. V~češtině se používají dvojité \uv{uvozovky} namísto anglických dvojitých
``uvozovek''. Ve smíšené sazbě se řez uvozovek řídí řezem prvního uvozovaného slova. Pokud je uvozována celá
věta, sází se koncová tečka před uvozovku, pokud se uvozuje slovo nebo část věty, sází se tečka za uvozovku.

V~originálním \LaTeX{u} bychom české uvozovky sázeli příliš pracně, proto je v~balíku czech k~dispozici příkaz
\verb!\uv{text}!, který je vysází podle českých zvyklostí.

Druhou odlišností je pravidlo pro sázení konců řádků. V~české sazbě by řádek neměl končit osamocenou
jednopísmennou předložkou nebo spojkou (spojkou~a končit může při sazbě do 25~liter). Abychom cs\LaTeX{u}
zabránili v~sázení osamocených předložek, vkládáme mezi předložku a slovo nezlomitelnou mezeru znakem \verb!~!
(vlnka, tilda). Pro automatické doplnění nezlomitelných mezer slouží volně šiřitelný program \textit{vlna}.

\end{document}

% End of file
