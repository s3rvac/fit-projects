\section{Práce v týmu a komunikace}

Každé úterý jsme se scházeli, abychom mohli prodiskutovat návrh programu a jeho jednotlivých částí, o přestávce na přednáškách jsme pak projednávali implementační detaily a procházeli specifikace v zadání.

Internetová komunikace (ICQ/IRC/Jabber) byla také důležitá, protože jsme pracovali v odlišných lokalitách, nebyla ústní domluva možná. Ke správě zdrojových souborů jsme využívali SVN\footnote{http://subversion.tigris.org/} repositář na serveru merlin, mohli jsme tak souběžně pracovat na zdrojových souborech prakticky bez omezení a urychlovalo to aktualizace a opravy chyb.

Byl také zaveden tzv. \uv{kódovací standard} - specifikace formátování k zachování konzistence pojmenování proměnných, odsazování atd. Během vývoje jsme se navzájem ihned upozorňovali, pokud jsme v programu nalezli chybu a informovali správce daného modulu - při drobnější chybě ji stačilo opravit přímo.
