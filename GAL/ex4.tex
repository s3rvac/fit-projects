%%%%%%%%%%%%%%%%%%%%%%%%%%%%%%%%%%%%%%%%%%%%%%%%%%%%%%%%%%%%%%%%%%%%%%%%%%%%%%%
% Project: GAL 2009
% Authors:
%     Radim Kapavik, xkapav01@stud.fit.vutbr.cz
%     Ondrej Lengal, xlenga00@stud.fit.vutbr.cz
%     Vojtech Storek, xstore02@stud.fit.vutbr.cz
%     Vit Triska, xtrisk01@stud.fit.vutbr.cz
%     Petr Zemek, xzemek02@stud.fit.vutbr.cz
%%%%%%%%%%%%%%%%%%%%%%%%%%%%%%%%%%%%%%%%%%%%%%%%%%%%%%%%%%%%%%%%%%%%%%%%%%%%%%%

\section*{Exercise 4}
\label{sec:Ex4}

\subsection*{Assignment}

Given an adjacency-list representation of a multigraph $G = (V, E)$, describe
an $O(|V| + |E|)$-time algorithm to compute the adjacency-list representation
of the ``equivalent'' undirected graph $G^{'} = (V, E^{'})$, where $E^{'}$
consists of the edges in $E$ with all multiple edges between two vertices
replaced by a single edge and with all self-loops removed.

\subsection*{Solution}

For a graph $G = (V, E)$, let $n = |V|$ and $m = |E|$.

\textbf{Main idea}

Let $G = (V, E)$ be a multigraph and $G^{'} = (V, E^{'})$ its ``equivalent''
undirected graph (as defined in the assignment). First, for every directed edge
$(u, v) \in E$, there must be an undirected edge in adjacency lists for both
vertices in $G^{'}$ ($u$ is connected to $v$ and $v$ is connected to $u$), so
$u$ will be appended to the (temporary --- see the following step) adjacency
list of $v$ and vice versa. While going through all edges and performing this
``direction removal'', loops can be taken out (for each vertex $u \in V$, do
not add $u$ to the adjacency list of $u$).

However, after this is done, adjacency-lists may contain duplicates, which must
be removed (there can be at most one edge between each two vertices). To remove
them and preserve the time complexity of the algorithm, the following procedure
is used. First, note that edges were stored into temporary adjacency lists in
the previous step and these temporary adjacency lists will now be filtered into
resulting adjacency lists. Second, initialize a temporary array $WasConnTo$ of
length $n$ (set values on all indices to $null$). Then, when removing
duplicities from the temporary adjacency list for vertex $u$, $WasConnTo[v] =
u$ means that vertex $v$ was already added to the adjacency list of $u$ (is
already connected to $u$) and $WasConnTo[v] \neq u$ means that $v$ was not
added to the adjacency list of $u$ (is not connected to $u$). After vertex $v$
is added to the adjacency list of $u$, set $WasConnTo[v] = u$ to assure
uniqueness of the edge.

After all duplicities are removed from the temporary adjacency-list of vertex
$u$, continue to remove duplicities for the next vertex $w$. Note that one does
not need to re-initialize $WasConnTo$ at this point, because one adjacency list
for each vertex is passed at a time, so now $WasConnTo[x] \neq w$ for all $x
\in V$ (means that $w$ is not connected to any vertex, which corresponds to the
fact that $Adj^{'}$ is initially empty for all vertices).

\textbf{Algorithm}

\begin{algorithm}[H]
	\KwIn{Adjacency-list representation of a multigraph $G = (V, E)$ ($Adj[u]$
	for each $u \in V$).}
	\KwOut{Adjacency-list representation of a graph $G^{'} = (V, E^{'})$, which
	has all multiple edges between two vertices replaced by a single edge and
	with all self-loops removed ($Adj^{'}[u]$ for each $u \in V$).}

	\tcp{Initialization}
	\ForEach{$u \in V$}{
		$WasConnTo[u] := null$\;
		$TmpAdj^{'}[u] := []$\;
		$Adj^{'}[u] := []$\;
	}
	\tcp{Remove loops and edge orientation}
	\tcp{(store edges into the temporary adjacency-list $TmpAdj^{'}$)}
	\ForEach{$u \in V$}{
		\ForEach{$v \in Adj[u]$}{
			\If(\tcp*[h]{do not include loops}){$u \neq v$}{
				\tcp{Create an undirected edge for both $u$ and $v$}
				$\mathrm{append}(TmpAdj^{'}[u], v)$\;
				$\mathrm{append}(TmpAdj^{'}[v], u)$\;
			}
		}
	}
	\tcp{Remove duplicities}
	\tcp{(store edges into the resulting adjacency-list $Adj^{'}$)}
	\ForEach{$u \in V$}{
		\ForEach{$v \in TmpAdj^{'}[u]$}{
			\If{$WasConnTo[v] \neq u$}{
				$WasConnTo[v] := u$\;
				$\mathrm{append}(Adj^{'}[u], v)$\;
			}
		}
	}
\end{algorithm}

\textbf{Complexity analysis}

The initialization part takes $n$ steps. The second part of the algorithm
(loops and edge orientation removal) needs $m$ steps (every edge must be
passed). Again, we assume that appending to a list is a $O(1)$ operation (see
\nameref{sec:Ex3}). Finally, to remove duplicities from the temporary
adjacency-list, one needs at most $2m$ steps (every edge, except loops, was
replaced with two undirected edges). This gives us the total time complexity
$O(n + m) + O(2m) = O(n + 3m) = O(n + m)$.

%%%%%%%%%%%%%%%%%%%%%%%%%%%%%%%%%%%%%%%%%%%%%%%%%%%%%%%%%%%%%%%%%%%%%%%%%%%%%%%
% vim: syntax=tex
%%%%%%%%%%%%%%%%%%%%%%%%%%%%%%%%%%%%%%%%%%%%%%%%%%%%%%%%%%%%%%%%%%%%%%%%%%%%%%%
